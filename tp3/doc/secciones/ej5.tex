\section{Ejercicio 5}
Como ya definimos anteriormente (\ref{rutinas}), si bien seteamos rutinas 'default' para excepciones, las rutinas de atención de interrupciones de teclado y reloj son diferentes.
Para estas interrupciones (controladas por el PIC) interactuamos con puertos I/O para luego hacer llamados a rutinas específicas ($rtc\_isr$, $keyboard\_isr$, etc) que respondan a estas interrupciones.
\\Por lo tanto lo único que haremos dentro de las $\_isr(\# int)$ definidas en $isr.asm$ será comunicar el fin de interrupción a los puertos del pic y hacer llamados a tareas correspondientes codeadas en C.
\\Caso aparte es el de la int 0x66, que es la que en la lógica del juego permitirá hacer las syscalls para llamar a $DONDE$, $SOY$ y $MAPEAR$ según un parámetro input en $eax$.

\subsection{Habilitar interrupciones de reloj}
Contamos con dos rutinas de atención a dos relojes diferentes: el RTC \footnote{Real-time clock } y el PIT \footnote{Programmable Interval Timer}, asignados a interrupciones nº 32 y 40 respectivamente. \\
Si bien en el contexto del uso dado en este sistema no presentan diferencias ambos relojes, delegamos en el RTC las cuestiones referentes al scheduler y dedicamos el PIT exclusivamente para el funcionamiento del audio \ref{audio}.
\\
\\
Para el RTC se corresponde la siguiente rutina:

\begin{lstlisting}
  void rtc_isr() {
      // Read RTC_C to ack the interruption
      outb(RTC_CMD, RTC_MASK_NMI | RTC_C);
      inb(RTC_DATA);

      game_tick();
  }
\end{lstlisting}

Donde la primer línea se encarga de reconocer la interrupción y se deshabilitan las NMI
\footnote{Non-maskable interrupt, \url{http://wiki.osdev.org/RTC\#Avoiding_NMI_and_Other_Interrupts_While_Programming}} mientras se accede el registro C del RTC (de no suceder esto, frente a determinadas interrupciones,
\href{http://wiki.osdev.org/RTC\#Interrupts_and_Register_C}{se podría deshabilitar la interrupción del RTC}). El acknowledge se completa leyendo el puerto de datos del RTC en la segunda línea. \\
Luego se hace un llamado a $game\_tick()$ (función de $game.c$) que llama a actualizarse al scheduler.



\subsection{Habilitar interrupciones de teclado}
Se lee del puerto $0x60$ un scan code que luego convertimos a ASCII usando un switch statement para cada botón de la interfaz del juego (ignorando los break codes, de modo que solo se acciona una interrupción al pulsar cada tecla).
A partir de esta conversión se determina si se actualiza la posición de un cursor, si se muestra la pantalla de restart, ayuda o se setea el debug mode \ref{isr_dbg} (entre otros).

\begin{lstlisting}
  unsigned char status2ASCII(unsigned char input){
      unsigned char output = 0;
      switch (input){
          case 0x01: // Esc
              output = 3;
              break;
          //Player 1
          case 0x11:
              output = 'W';
              break;
          case 0x12:
              output = 'E';
  // [...]
  }
\end{lstlisting}


\subsection{Debugger}
\label{isr_dbg}
Dentro de la rutina de $keyboard\_isr$, si el debugger (\ref{dbg}) se encuentra en pantalla entonces solamente se atenderá la interrupción de teclado que lo desactiva (correspondiente a la letra 'Y').
Por lo tanto no se podrán mover cursores cuando se esté mostrando.\\
Para habilitar el modo debug se llama a la interrupción de teclado de la letra 'Y' y se habilita un flag $dbg\_enabled$ desde la función $game\_enable\_debugger$ antes de volver a la ejecución de la tarea. La próxima vez que se produzca una excepción o un error de parámetros en syscall ($game\_kill\_task$), se detendrá la ejecución y se llamará al debugger hasta que se presione nuevamente la tecla 'Y' y se salte a la tarea idle ($game\_go\_idle$) para retomar la ejecución del juego.

\subsection{Habilitar interrupción $0x66$}
Se define para la $int\ 0x66$ una syscall, cuyo funcionamiento 'ingame' detallaremos más adelante (\ref{syscalls} ).
Al entrar en la isr correspondiente se produce un llamado a funciones externas de $game.c$ en función de el parámetro pasado en $eax$, si este parámetro no es válido se procede a terminar la tarea que hizo la llamada a sistema por prevención de que esté corrupta.
\begin{lstlisting}
  global _isr0x66
  _isr0x66:
      pushad

      cmp eax, SYSCALL_DONDE
      jne .not_donde
          push ebx
          call game_donde
          add esp, 4
          jmp .end
      .not_donde:
  ;  [...]

  ; Syscall invalido
  call game_kill_task

  .end:
  popad
  iret

\end{lstlisting}
