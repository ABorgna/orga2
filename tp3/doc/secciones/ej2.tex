\subsection{Ejercicio 2}

La IDT se corresponde con un arreglo de tamaño 255 de idt_entry. Un idt_entry es un struct que representará cada gate dentro de la tabla indicando selector de segmento de código donde se almacena la ISR, offset en dicho segmento y atributos del descriptor (present gate, tipo de gate y dpl). 

\begin{lstlisting} [caption={Struct de las gates},label=gate-struct]
typedef struct str_idt_entry_fld {
    unsigned short offset_0_15;
    unsigned short segsel;
    unsigned short attr;
    unsigned short offset_16_31;
} __attribute__((__packed__, aligned (8))) idt_entry;
\end{lstlisting}

\begin{figure}[H]
    \centering
    \includegraphics[width=\textwidth]{gates}
    \caption{Formato de las gates en IDT}
    \label{fig:gates}
\end{figure}



\subsubsection{Inicializar la IDT}

Para inicializar la IDT usamos una serie de macros tanto en C (para las gates) como en ASM (para las ISR) para facilitar la definición de las mismas.
De los macros generados usamos particularmente 3 de estos.
\begin{description}
\item [IDT_ENTRY_INTERRUPT] para las interrupciones de tipo 0-19, 32,33 y 40. Se setea con el descriptor de nuestro segmento flat de código de nivel 0, y offset (dividido de 0 a 15 y 16 a 31 bits por cuestiones de arquitectura) con puntero a una rutina genérica del 0-19, y específica para reloj (PIT y RTC) y teclado en interrupiones 32, 40 y 33 respectivamente. \\
Los atributos corresponden a nivel de privilegio 0, bit de present en 1 y tipo = 0xE >> 2
\item [IDT_ENTRY_DEFAULT] para las interrupciones reservadas por Intel según manual (20-31) y aquellas "libres" para el usuario que no definimos (por lo tanto, no deberían ser accedidas durante la ejecución del sistema). 
\item [IDT_ENTRY_INTERRUPT_USER] para syscalls int 0x66 con privilegios de nivel usuario para llamar a SOY, DONDE y MAPEAR.
\end{description} 

Originalmente las interrupciones no-default se encargan de imprimir por pantalla su número de interrupción con un define que se hace dentro del macro de la isr.
\begin{lstlisting} [caption={Mensaje en el macro de isr},label=macro-msg]
%macro ISR 2
global _isr%1

interrupt_msg_%1 db         %2
interrupt_msg_%1_len equ    $ \$ $ - interrupt_msg_%1

\end{lstlisting}

Todo esto quedó encapsulado en una función llamada idt_inicializar() que es ejecutada luego de cargar en IDTR un descriptor con la dirección donde se encuentra la IDT y su límite, empaquetados en el struct idt_descriptor de la cátedra.


\subsubsection{Habilitar la IDT}
