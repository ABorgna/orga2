\section{Apéndice - Audio}

La mayoría de las computadoras poseen un speaker conectado como periférico
al procesador que permite generar sonidos de alerta.

Este speaker está directamente conectado al puerto 2 del Programmable Interval Timer (PIT).
Usando el modo generador de ondas cuadradas del PIT podemos entonces generar una señal de salida audible. \\
Ya que esta señal es siempre un a onda cuadrada con duty 50\% la calidad del sonido es bastante mala,
pero pudimos usarla para reproducir pistas de audio de dos canales.

Bochs nos permite emular el speaker habilitando el plugin de la soundblaster16, si se agrega el flag $--enable-sb16$ al compilar.

\subsection{Configuración de los periféricos}

Habilitamos la conexión del PIT con el speaker seteando los bits 1 y 2 del puerto $0x61$ de IO.

\begin{lstlisting}
tmp = inb(0x61);
if (tmp != (tmp | 3)) {
    outb(0x61, tmp | 3);
}
\end{lstlisting}

Y para configurar el PIT lo seteamos en modo 3 (square wave) y le enviamos un valor de contador.
Este valor lo usará para dividir su frequencia base de 1.19MHZ, para obtener como resultado una frecuencia audible.

\begin{lstlisting}
// Configure the pit, mode 3 (square wave)
outb(0x43, 0x36 | (channel << 6));

// Send the 16b count to the corresponding port
outb(0x40 + channel, (uint8_t) (divisor & 0xFF));
outb(0x40 + channel, (uint8_t) (divisor >> 8));
\end{lstlisting}

\subsection{Reproducción de MIDIs}

Diseñamos un formato compacto de archivo de audio compuesto por secuencias de bloques de dos bytes.
El primer byte es el valor de una nota MIDI, un entero entre 1 y 127 (o 0 si es silencio) que se
traduce a un valor de frecuencia con la siguiente fórmula:

$$ freq = 440 * 2^{\frac{nota - 69}{12}} Hz $$

Y el segundo byte es la duración en $ms$.

Desarrollamos un conversor de archivos MIDI a nuestro formato (src/tool/frommidi.py),
que genera archivos binarios $.audio$ , y usando la pseudoinstruccion $incbin$ de NASM
los incluimos en el código del kernel.

