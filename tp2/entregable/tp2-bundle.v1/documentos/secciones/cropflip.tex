\section{Cropflip}

\subsection{Descripción}
La imagen destino del filtro consiste en invertir verticalmente (flip) un recorte (crop) de la imagen fuente a partir de offsets dados como parámetro. El ancho y alto en píxeles de la imagen destino también se pasa como parámetros.
La descripción matemática está dada por la fórmula:

$$ O_{i,\ j}^{k}=I_{tamy+offsety-i-1, \ \ offsetx+j}^{k} $$

Donde el 'crop' de la imagen input corresponde con los píxeles del tipo:

$$
I_{offsety+i, \ \ offsetx+j}^{k} 
\qquad \text{con} \quad 0 \leq i < tamy \ \ \ 0 \leq j < tamx 
$$

\begin{table}[h]
\centering
\mem
\begin{tabular}{l|c|c|c|c|c|c|l}
 & \multicolumn{1}{l|}{}      & \multicolumn{1}{l|}{}       & \multicolumn{1}{l|}{}       & \multicolumn{1}{l|}{}       & \multicolumn{1}{l|}{}       & \multicolumn{1}{l|}{}      &  \\ \hline
 & \cellcolor[HTML]{FFCB2F}$I_{30}$ & \cellcolor[HTML]{FFCB2F}$I_{31}$  & \cellcolor[HTML]{FFCB2F}$I_{32}$  & \cellcolor[HTML]{FD6864}$I_{33}$  & \cellcolor[HTML]{FD6864}$I_{34}$  & \cellcolor[HTML]{FD6864}$I_{35}$ &  \\ \hline
 & \cellcolor[HTML]{FFCB2F}$I_{24}$ & \cellcolor[HTML]{FFCB2F}$I_{25}$  & \cellcolor[HTML]{FFCB2F}$I_{26}$  & \cellcolor[HTML]{FD6864}$I_{27}$  & \cellcolor[HTML]{FD6864}$I_{28}$  & \cellcolor[HTML]{FD6864}$I_{29}$ &  \\ \hline
 & \cellcolor[HTML]{FFCB2F}$I_{18}$ & \cellcolor[HTML]{FFCB2F}$I_{19}$ & \cellcolor[HTML]{FFCB2F}$I_{20}$ & \cellcolor[HTML]{FD6864}$I_{21}$ & \cellcolor[HTML]{FD6864}$I_{22}$  & \cellcolor[HTML]{FD6864}$I_{23}$ &  \\ \hline
 & \cellcolor[HTML]{FFCB2F}$I_{12}$ & \cellcolor[HTML]{FFCB2F}$I_{13}$ & \cellcolor[HTML]{FFCB2F}$I_{14}$ & \cellcolor[HTML]{FD6864}$I_{15}$ & \cellcolor[HTML]{FD6864}$I_{16}$  & \cellcolor[HTML]{FD6864}$I_{17}$ &  \\ \hline
 & \cellcolor[HTML]{FFCB2F}$I_{6}$ & \cellcolor[HTML]{FFCB2F}$I_{7}$ & \cellcolor[HTML]{FFCB2F}$I_{8}$ & \cellcolor[HTML]{FFCB2F}$I_{9}$ & \cellcolor[HTML]{FFCB2F}$I_{10}$  & \cellcolor[HTML]{FFCB2F}$I_{11}$ &  \\ \hline
 & \cellcolor[HTML]{FFCB2F}$I_{0}$ & \cellcolor[HTML]{FFCB2F}$I_{1}$ & \cellcolor[HTML]{FFCB2F}$I_{2}$ & \cellcolor[HTML]{FFCB2F}$I_{3}$ & \cellcolor[HTML]{FFCB2F}$I_{4}$  & \cellcolor[HTML]{FFCB2F}$I_{5}$ &  \\ \hline
 & \multicolumn{1}{l|}{}      & \multicolumn{1}{l|}{}       & \multicolumn{1}{l|}{}       & \multicolumn{1}{l|}{}       & \multicolumn{1}{l|}{}       & \multicolumn{1}{l|}{}      &
\end{tabular}
\caption{Ilustracion de la imagen fuente en memoria. En rojo los pixeles del crop \newline
(offsetx = 3, offsety = 2, tamx = 4, tamy = 4)}
\end{table}

\begin{table}[h]
\centering
\mem
\begin{tabular}{l|c|c|c|l}
& \multicolumn{1}{l|}{}       & \multicolumn{1}{l|}{}       & \multicolumn{1}{l|}{}      &  \\ \hline
 & \cellcolor[HTML]{FD6864}$I_{15}$ & \cellcolor[HTML]{FD6864}$I_{16}$  & \cellcolor[HTML]{FD6864}$I_{17}$ &  \\ \hline
 & \cellcolor[HTML]{FD6864}$I_{21}$ & \cellcolor[HTML]{FD6864}$I_{22}$  & \cellcolor[HTML]{FD6864}$I_{23}$ &  \\ \hline 
   & \cellcolor[HTML]{FD6864}$I_{27}$  & \cellcolor[HTML]{FD6864}$I_{28}$  & \cellcolor[HTML]{FD6864}$I_{29}$ &  \\ \hline
 & \cellcolor[HTML]{FD6864}$I_{33}$  & \cellcolor[HTML]{FD6864}$I_{34}$  & \cellcolor[HTML]{FD6864}$I_{35}$ &  \\ \hline
  & \multicolumn{1}{l|}{}       & \multicolumn{1}{l|}{}       & \multicolumn{1}{l|}{}      &
\end{tabular}
\caption{Ilustracion de la imagen destino en memoria}
\end{table}




No es dificil notar que las filas del crop en la fuente no constituyen una tira contigua de píxeles en memoria, si no que se encuentran distanciadas por el tamaño en bytes de offsetx

\subsection{Implementaciones}
Al no involucrar operaciones aritméticas entre componentes de la imagen, las implementaciones del filtro se centran en accesos a memoria. Por lo tanto, las implementaciones solamente difieren en el modo en que se copian y asignan píxeles a la imagen output. 
\subsubsection{Implementaciones C y SSE}
La implementación C simplemente se trata de recorrer la imagen con una sola variable aplicando pixel a pixel la transformación dada por la fórmula matemática previamente mencionada. Mientras que la implementación corresponiente a SSE recorre la imagen fuente desde la fila superior del recuadro del crop hacia abajo de a 4 píxeles por iteración como indica el siguiente pseudo-código: 


\begin{codesnippet}
\begin{verbatim}

temp = src + offsetx*4 +srcRowSize*(offsety+tamy-1)       
{Apunta al primer píxel de la esquina superior izq. del crop}
for i = 0 to tamy:  
  for j = 0 to tamx:  
    xmm0 = [temp]
    [dst] = xmm0 
    dst = dst  + 16 
    temp =  temp + 16 
  temp = temp - (dstRowSize + srcRowSize)       
	{Decrece el ancho de la imagen destino y el de la fuente para apuntar primer pixel de la fila de
 abajo a la que procesó}    
  end for 
end for 

\end{verbatim}
\end{codesnippet}



De esta manera, las lecturas y escrituras en memoria representan $\frac{1}{16} = 0,0625 =  6,25\%$ del total de los accesos a memoria de la implementación C.


\subsubsection{Implementaciones SIMD paralelo en 128 y 256 bits}
Se diferencian de la implementación anterior de SSE en el uso de la mayor cantidad posible de registros XMM (YMM de 256 bits en el caso de las implementaciones de AVX2) para las operaciones de transferencia de bloques de pixeles de la imagen fuente a la destino. 
\\

Dado que las lecturas y escrituras en memoria de cada registro son independientes entre sí, la ejecución fuera de orden del procesador hace que la transferencia por bloque sea más rápida que la implementación individual (que recorre la imagen con un único registro XMM). Por cuestiones de vecindad espacial de los píxeles en la memoria, los accesos a memoria de cada registro tienen chances particularmente altas de hitrate en caché, no demorando así las lecturas del resto de los registros.
\\

Recordando que los registros de AVX2 tienen 32 bytes de capacidad y cada píxel en nuestro formato ocupa 4 bytes, cada registro YMM tendrá capacidad para:

$$ \frac{32 \ bytes}{4 \ \frac{bytes}{px}} = 8 \ px $$

Por lo cual, suponiendo $ tamx \equiv 0 \ (mod \ 8) $ (como es el caso de una imagen de salida de 512x512), tendríamos $\frac{1}{32} = 0,03125 =  3,125\%$ del total de accesos a memoria de la implementación en C.

\subsubsection{Copiado paralelo de vectores}

La asignación de píxeles se hace llamando a las funciones externas 'copyN_sse' y 'copyN_avx2'\footnote{../entregable/tp2-bundle.v1/codigo/lib} que copian tiras de píxeles de una imagen a otra por bloques de 64/4/1 ó 128/8/1 píxeles (sse y avx2 respectivamente) según sea posible. A modo de ejemplo (los otros casos son análogos), para copiar bloques de 64px (256B) desde la posición indicada por rsi a la indicada por rdi se cargan los valores correspondientes de la siguiente manera:
\newline
\\
xmm0 $\leftarrow$ {[rsi]} \\
xmm1 $\leftarrow$ {[rsi+16]} \\
... \\
xmm14 $\leftarrow$ {[rsi+224]}  \\
xmm15 $\leftarrow$ {[rsi+240]} \\
\\
{[rdi]} $\leftarrow$ xmm0 \\
{[rdi+16]} $\leftarrow$ xmm1 \\
... \\
{[rdi+224]} $\leftarrow$ xmm14 \\
{[rdi+240]} $\leftarrow$ xmm15 \\

En el caso del filtro Cropflip, las tiras de píxeles corresponden a tiras de tamaño 'tamx' (es decir, al ancho del crop).  


\subsection{Rendimiento y análisis}

A diferencia de los otros dos filtros, donde el procesamiento de imágenes involucra operaciones aritméticas paralelas, la implementación AVX2 de Cropflip no presenta un 'speedup' tan significativo respecto de las implementaciones de SSE con registros de 128 bits en paralelo.

