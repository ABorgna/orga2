\section{Sepia}

\subsection{Descripción}

El filtro sepia consiste en cambiar los colores de cada pixel de la siguiente manera:

$$ O^r_{i,j} = 0,5 \cdot suma_{i,j} $$
$$ O^g_{i,j} = 0,3 \cdot suma_{i,j} $$
$$ O^b_{i,j} = 0,2 \cdot suma_{i,j} $$
$$ O^a_{i,j} = I^a_{i,j} $$

donde $suma_{i,j} = I^r_{i,j} + I^g_{i,j} + I^b_{i,j}$

Como puede verse, este filtro no recibe ningún parámetro y trabaja directamente con los datos mismos de la imagen.

\subsection{Implementaciones}

Antes de entrar en detalle a cada una de las implementaciones hechas, es importante notar que

$$suma_{i,j} = I^r_{i,j} + I^g_{i,j} + I^b_{i,j} \leq 255 + 255 + 255 = 3 \cdot 255$$

Lo cual excede el rango de representación de las componentes de un píxel. Esto puede resultar en un problema o no dependiendo de cada caso, ya que es necesario calcular $suma_{i,j}$ para resultados intermedios. Como el resultado final de aplicar el filtro en cada componente es multiplicar $suma_{i,j}$ por un número menor o igual a 1, se consideró aplicar la ley distributiva para poder salvar este problema, ya que, por ejemplo, en el caso de la componente azul

$$ O^b_{i,j} = 0,2 \cdot suma_{i,j} = 0,2 \cdot (I^r_{i,j} + I^g_{i,j} + I^b_{i,j}) = 0,2 \cdot I^r_{i,j} +  0,2 \cdot I^g_{i,j} + 0,2 \cdot I^b_{i,j} \leq 0,2 \cdot 255 + 0,2 \cdot 255 + 0,2 \cdot 255 = 0,6 \cdot 255 \leq 255$$


y en el caso de la componente verde

$$ O^b_{i,j} = 0,3 \cdot suma_{i,j} = 0,3 \cdot (I^r_{i,j} + I^g_{i,j} + I^b_{i,j}) = 0,3 \cdot I^r_{i,j} +  0,3 \cdot I^g_{i,j} + 0,3 \cdot I^b_{i,j} \leq 0,3 \cdot 255 + 0,3 \cdot 255 + 0,3 \cdot 255 = 0,9 \cdot 255 \leq 255$$

Cabe resaltar que realizando las operaciones de este modo, nunca se llega a un resultado intermedio el cual exceda el límite de representación de las componentes.

Sin embargo, esta alternativa fue descartada ya que implica hacer el triple de multiplicaciones en punto flotante, lo cual reduce drásticamente la performance.

Además, se debe notar que en el caso de la componente roja ocurre lo siguiente:

	$$ O^b_{i,j} = 0,5 \cdot suma_{i,j} \leq 0,5 \cdot 3 \cdot 255 = 1,5 \cdot 255$$

Lo cual resulta un problema a la hora de almacenar el resultado final ya que el mismo podría exceder el rango de representación de la componente roja. En cada una de las implementaciones se detalla cómo fue resuelta esta cuestión.

\subsubsection{C}

A grandes rasgos, la implementación en C es intuitiva, ya que recorre cada uno de los píxeles uno por uno, por cada iteración crea una variable auxiliar llamada $suma$ (la cual, como puede esperarse, contiene el valor de $suma_{i,j}$) y asigna a cada componente el valor de dicha suma multiplicada por su respectivo factor.
Debido a las dos problemáticas que se plantearon anteriormente, se hicieron algunos ajustes con respecto a la implementación. La variable auxiliar $suma$ es una variable del tipo $unsigned short (16 bits)$  para no perder precisión en las operaciones. Por la misma razón, se creó también una nueva variable auxiliar $suma_r$, la cual contiene el valor de $0,5 \cdot suma_{i,j} = I^r_{i,j}$.
Para volver a $char (8 bits)$ se resuelve distinto en cada caso.
\begin{itemize}
	\item En el caso de $suma$ con respecto a las componentes azul y verde, solamente se les asigna a cada una el valor de $suma$ multiplicado por su respectivo factor ya que, como se demostró anteriormente, $O^b_{i,j}$ y $O^g_{i,j}$ son siempre menores a 255 y por lo tanto no hay que tener ningún recaudo extra. Luego de ejecutar la multiplicación, el lenguaje C simplemente convierte el dato a $char$ y lo asigna.
	\item En el caso de $suma_r$, se pregunta primero si el resultado final dio mayor o igual a 255. Si eso es cierto, lo satura a 255. Caso contrario, lo deja como está. En esencia se está ejecutando un $min(I^r_{i,j},255)$.
\end{itemize}

La cantidad de operaciones de punto flotante por píxel en esta implementación es de $\frac{3op}{p}$.

\subsubsection{Asm - SSE}

Cada componente ocupa 1 byte y cada píxel tiene 4 componentes, por lo cual un píxel ocupa 4 bytes. Como los registros xmm son de 16B, es posible procesar 4 píxeles en un solo registro simultáneamente.
Al inicio del loop, se copian 4 píxeles de la imagen fuente a xmm0 para empezar a procesar. Para simplificar la notación, se llamará $K_{i}$ a la componente K del pixel $i \in {0,1,2,3}$

\xmm{0} \xmmByte{$A_{3}$}{$R_{3}$}{$G_{3}$}{$B_{3}$}{$A_{2}$}{$R_{2}$}{$G_{2}$}{$B_{2}$}{$A_{1}$}{$R_{1}$}{$G_{1}$}{$B_{1}$}{$A_{0}$}{$R_{0}$}{$G_{0}$}{$B_{0}$}

Luego, se copia a xmm1 y xmm3 el contenido de xmm0 y se borran los alpha de los mismos

\xmm{1} \xmmByte{$0$}{$R_{3}$}{$G_{3}$}{$B_{3}$}{$0$}{$R_{2}$}{$G_{2}$}{$B_{2}$}{$0$}{$R_{1}$}{$G_{1}$}{$B_{1}$}{$0$}{$R_{0}$}{$G_{0}$}{$B_{0}$}

\xmm{3} \xmmByte{$0$}{$R_{3}$}{$G_{3}$}{$B_{3}$}{$0$}{$R_{2}$}{$G_{2}$}{$B_{2}$}{$0$}{$R_{1}$}{$G_{1}$}{$B_{1}$}{$0$}{$R_{0}$}{$G_{0}$}{$B_{0}$}

Se desempaquetan xmm1 y xmm3 de tal manera que xmm1 tenga la parte baja y xmm3 la parte alta. Ambos terminarían convirtiéndose en registros de words empaquetados. Esto se hace con el fin de poder luego ejecutar una suma horizontal de a word sin el riesgo de que el resultado quede fuera del rango de representación.

\xmm{1} \xmmWord{$0$}{$R_{1}$}{$G_{1}$}{$B_{1}$}{$0$}{$R_{0}$}{$G_{0}$}{$B_{0}$}

\xmm{3} \xmmWord{$0$}{$R_{3}$}{$G_{3}$}{$B_{3}$}{$0$}{$R_{2}$}{$G_{2}$}{$B_{2}$}

Se ejecutan las sumas horizontales necesarias para que xmm1 tenga el resultado

\xmm{1} \xmmWord{$0$}{$0$}{$0$}{$0$}{$S_{3}$}{$S_{2}$}{$S_{1}$}{$S_{0}$}

donde $S_{i} = R_{i} + G_{i} + B_{i}$

Se convierte xmm1 a float para poder realizar la multiplicación por los factores correspondientes, y luego se ejecutan varios shuffle de tal manera que haya un registro completo por cada píxel

\xmm{1} \xmmFloat{$S_{0}$}{$S_{0}$}{$S_{0}$}{$S_{0}$}

\xmm{2} \xmmFloat{$S_{1}$}{$S_{1}$}{$S_{1}$}{$S_{1}$}

\xmm{3} \xmmFloat{$S_{2}$}{$S_{2}$}{$S_{2}$}{$S_{2}$}

\xmm{4} \xmmFloat{$S_{3}$}{$S_{3}$}{$S_{3}$}{$S_{3}$}

Se multiplica a los cuatro registros por el siguiente vector de factores que se encuentra alojado en xmm15

\xmm{15} \xmmFloat{$0$}{$0,5$}{$0,3$}{$0,2$}

Lo cual da como resultado

\xmm{1} \xmmFloat{$0$}{$0,5 \cdot S_{0}$}{$0,3 \cdot S_{0}$}{$0,2 \cdot S_{0}$}

\xmm{2} \xmmFloat{$0$}{$0,5 \cdot S_{1}$}{$0,3 \cdot S_{1}$}{$0,2 \cdot S_{1}$}

\xmm{3} \xmmFloat{$0$}{$0,5 \cdot S_{2}$}{$0,3 \cdot S_{2}$}{$0,2 \cdot S_{2}$}

\xmm{4} \xmmFloat{$0$}{$0,5 \cdot S_{3}$}{$0,3 \cdot S_{3}$}{$0,2 \cdot S_{3}$}

Es decir

\xmm{1} \xmmFloat{$0$}{$O^r_{0}$}{$O^g_{0}$}{$O^b_{0}$}

\xmm{2} \xmmFloat{$0$}{$O^r_{1}$}{$O^g_{1}$}{$O^b_{1}$}

\xmm{3} \xmmFloat{$0$}{$O^r_{2}$}{$O^g_{2}$}{$O^b_{2}$}

\xmm{4} \xmmFloat{$0$}{$O^r_{3}$}{$O^g_{3}$}{$O^b_{3}$}

Se reconvierte cada registro a int y se empaquetan de manera saturada los datos en xmm1. La saturación es la solución al overflow de la componente roja, ya que la instrucción misma la transforma en 255 en caso de ser mayor.

\xmm{1} \xmmByte{$0$}{$O^r_{3}$}{$O^g_{3}$}{$O^b_{3}$}{$0$}{$O^r_{2}$}{$O^g_{2}$}{$O^b_{2}$}{$0$}{$O^r_{1}$}{$O^g_{1}$}{$O^b_{1}$}{$0$}{$O^r_{0}$}{$O^g_{0}$}{$O^b_{0}$}

Por el lado de xmm0, se eliminan los datos de los colores y se deja solo el alpha para poder luego "fusionar" los datos con xmm1

\xmm{0} \xmmByte{$A_{3}$}{$0$}{$0$}{$0$}{$A_{2}$}{$0$}{$0$}{$0$}{$A_{1}$}{$0$}{$0$}{$0$}{$A_{0}$}{$0$}{$0$}{$0$}

Por último, se unen los datos de xmm0 y xmm1 y se almacena en la imagen destino.

\xmm{0} \xmmByte{$O^a_{3}$}{$O^r_{3}$}{$O^g_{3}$}{$O^b_{3}$}{$O^a_{2}$}{$O^r_{2}$}{$O^g_{2}$}{$O^b_{2}$}{$O^a_{1}$}{$O^r_{1}$}{$O^g_{1}$}{$O^b_{1}$}{$O^a_{0}$}{$O^r_{0}$}{$O^g_{0}$}{$O^b_{0}$}

La cantidad de operaciones de punto flotante por píxel en esta implementación es de $\frac{4op}{4p} = \frac{1op}{p}$, La cual es menor comparado a la cantidad de operaciones por píxel de la implementación en C (la cual es $\frac{3op}{p}$).

Con la finalidad de aumentar el rendimiento de esta implementación, se decidió utilizar más registros para procesar más cantidad de memoria en un mismo loop, de modo tal que la ejecución fuera de orden fuera más efectiva y se realice la menor cantidad de saltos condicionales posible. Como se pudo ver en el desarrollo del algoritmo, solo cinco registros fueron utilizados para realizar todo el proceso (xmm0..xmm4), por lo cual, en vez de procesar de a 4 píxeles, el ciclo ahora procesa de a 12, utilizando 15 de los 16 registros xmm, dejando espacio libre para el registro extra que contiene los factores por los que hay que multiplicar. También es necesario utilizar un registro lleno de ceros para desempaquetar los datos, lo cual haría necesario contar con 17 registros o acceder a memoria, pero como ese registro de ceros solo es necesario durante una sola parte del algoritmo (en la cual aún no están en uso todos los registros) se puede utilizar algún registro que aún no se haya procesado.
La problemática que surge a partir de procesar de a 12 píxeles, es que no se sabe si se puede estar procesando píxeles de más, ya que solo se tiene como hipótesis que el ancho de la imagen multiplicado por su altura es un número que es múltiplo de 8, y por ende no siempre tendremos una imagen cuyo tamaño total sea múltiplo de 12. Para solucionar esto, se procesa de a 12 píxeles todas las veces que se pueda, y cuando ya no se pueda más, se procesa de a 4 como originalmente se hacía, hasta terminar de procesar toda la imagen. Notar que no puede llegar a ocurrir que el algoritmo haya tenido que procesar de a 4 píxeles más de 2 veces.

\subsubsection{Asm - AVX2}

En AVX2 podemos utilizar los registros extendidos ymm de 256 bits, por lo cual podemos procesar el doble de píxeles que en la implementación SSE. También contamos con instrucciones no destructivas que nos ahorran un par de pasos a la hora de implementar.
La conversión del algoritmo es directa con respecto a SSE. La estrategia es la misma, con la salvedad de que hay que tener cuidado con algunas instrucciones que no funcionan exactamente igual a su instrucción análoga de SSE. Tales son los casos de VPHADD, VPUNPCKLBW y VPUNPCKHBW.

\subsection{Experimentos}

