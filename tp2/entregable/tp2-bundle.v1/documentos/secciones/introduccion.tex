\section{Introducción}

\subsection{Objetivos generales}

El objetivo de este trabajo práctico es experimentar con sets de instrucciones SIMD de Intel ASM x86 sobre una serie de implementaciones de filtros en imágenes de mapas de bits (RGBA) con el propósito de entender mejor el rol del paralelismo de datos en este tipo de implementaciones y qué tan remunerador es. 
A partir de cada filtro, se plantearán implementaciones en C, SSE \footnote{Streaming SIMD Extensions} (set de instrucciones sobre registros de 128 bits), AVX \footnote{Advanced Vector Extensions} (también sobre 128 bits pero permitiendo operaciones en formato 'no-destructivo' \footnote{Preservan los operandos fuente}) y su expansión de 256 bits AVX2, Acompañadas todas de un análisis para poder contrastar y discutir su rendimiento.

\subsection{Notación de filtros}

Dado que la temática de la aplicación de este trabajo práctico es el procesamiento de imágenes, es preciso especificar con qué tipo de imágenes vamos a trabajar y cómo vamos a describir matemáticamente las funciones que apliquen los filtros.

La imágenes sobre las que trabajamos son consideradas como matrices de píxeles. Cada pixel posee cuatro componentes: rojo (r), verde (g), azul (b) y transparencia (a). Cada una de estas componentes tiene un largo de 8 bits, por lo cual su rango de representación es $[0, 255]$.

Se define $I^k_{i,j}$ como la componente $k \in \{r,g,b,a\}$ en la fila $i$ y la columna $j$ de la imagen, donde $i$ crece de abajo hacia arriba y $j$ crece de izquierda a derecha. Por ejemplo, el pixel de la esquina inferior izquierda correspondería a la fila 0 y a la columna 0.

Se llama $O^k_{i,j}$ a la imagen de salida generada por el filtro especificado. Por ejemplo, el siguiente filtro asigna a todas las componentes de cada pixel el valor de la componente roja que se encuentra en su respectivo pixel.

$$\forall k \in \{r,g,b,a\} \qquad O^k_{i,j} = I^r_{i,j}$$

\subsection{Mediciones y gráficos}

Para realizar mediciones implementamos un script de benchmarking automático\footnote{codigo/benchmark/benchmark.py} que nos facilite la tarea y reduzca el error humano.

Para cada combinación de filtro, implementación, imagen, tamaño y parámetros adicionales corremos el tp2 un mínimo de 100 iteraciones y 2 segundos. Empezando con el mínimo de iteraciones realizamos una búsqueda exponencial hasta encontrar una cantidad que sobrepase el límite de tiempo. Usando la cantidad de iteraciones encontrada calculamos los percentiles y el promedio del tiempo con una precisión de 1uS y cantidad de ciclos por ejecución. Además, guardamos el porcentaje promedio de misses a la caché y el porcentaje promedio de mispredicciones de salto.

Nuestra medida de ciclos de ejecución es provista por la instrucción \asm{rdtsc}. En todos los procesadores donde realizamos las mediciones el \asm{tsc} se incrementa a una frecuencia fija e independiente, y se encuentra sincronizado entre núcleos del procesador\footnote{Se puede verificar esta característica buscando el flag $constant\_tsc$ en $/proc/cpuid$}\textsuperscript{\cite[Volume 3B, Chapter 17.15]{intelsys}}, por lo que resulta una medida válida de comparación entre ejecuciones en el mismo procesador.

Para las mediciones de hits a cache y misses del branch predictor utilizamos la herramienta $perf$\footnote{https://perf.wiki.kernel.org/index.php/Main_Page} que nos devuelve los valores totales sobre la ejecución del programa a travéz de los registros contadores de performance de la cpu.
\\

En los gráficos presentados usaremos siempre la media a menos que se indique lo contrario. Y en los gráficos de barras representamos además los percentiles 10 y 90 como lineas de error.
\\

\subsection{Aclaración sobre el redondeo de punto flotante}

Por defecto las conversiones de punto flotante a números enteros se realizan redondeando al número par mas cercano\textsuperscript{\cite[Volume 1, Chapter 10.2.3]{intelsys}}, mientras que las operaciones de división de enteros siempre redondean hacia abajo.
Como nuestros filtros, si usan operaciones de punto flotante, intentan emular operaciones de enteros nos conviene configurar el redondeo de flotantes hacia cero.

Esto se logra seteando los bits 13 y 14 del registro \asm{MXCSR} con instrucciones.
Como el registro es calee-saved deberemos guardarlo antes de modificarlo.
Le cargaremos el valor por defecto, 0x1F80, mas la máscara que queremos.

\begin{lstlisting}
; Guardamos el registro
SUB RSP, 8
STMXCSR [RSP]
; Cargamos nuestro valor
LDMXCSR [MXCSR_RZ] ; MXCSR_RZ tiene el valor x7F80
\end{lstlisting}

Y al finalizar la función volvemos a cargar el valor original.

Esto lo haremos en todas las implementaciones que trabajen con números flotantes.

